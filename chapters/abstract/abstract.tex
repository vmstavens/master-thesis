\setlength{\parindent}{0pt}
\normalsize

% \noindent Some abstract text explaining the goal, methods and conclusion of the project. \\

% This project presents a new \gls{pnp} pipeline by excluding the vision systems and using purely tactile inputs to eliminate the weaknesses of vision algorithms such as transparency and reflectance. To develop this pipeline three subproblems were identified: tactile perception, pose estimation and in-hand manipulation. Tactile perception consisted of estimating the contact point, normals and skew forces. Using physics engine assistance, all three quantities were succesfully estimated. The pose estimation was successfully achieved on synthetic data with an orientation error less than \SI{5}{\degree} and translation error less than \SI{1}{\centi\meter} at \SI{10}{\percent} outliers. The in-hand manipulation problem was successfully solved using \gls{dapg}, which entailed an analysis of the training process and end performance. Finally, potential improvements to the pipeline is discussed for future iterations.

In this project, a novel pipeline for a \gls{pnp} task is presented, focusing on utilizing tactile inputs instead of vision systems to overcome the limitations of vision algorithms, such as transparency and reflectance issues. The development of this pipeline involved addressing three key subproblems: \gls{tp}, \gls{pe}, and in-hand manipulation. \gls{tp} involved accurately estimating the contact points, contact normals, and skew forces. The contact normals were successfully estimated using \gls{rls}, the contact points were found using the grasp matrix, while a \gls{dl} model was attempted to be used for skew force estimation without success. To compensate for the missing skew forces and cases of contact normals assistance of the Gazebo physics engine was applied successfully. The \gls{pe} task demonstrated promising results on synthetic data using \gls{gnc} for outlier rejection and \gls{rcqp} for transformation estimation. This resulted in an orientation error of \num{3} degrees and a translation error of  \SI{0.08}{\centi\meter} while in the presence of \SI{10}{\percent} outliers, which was within the criteria for success. For the in-hand manipulation problem, the use of \gls{dapg} proved successful in the MuJoCo dynamic simulator, and a comprehensive analysis of the training process and final performance was conducted. Lastly, potential enhancements for future iterations of the pipeline are discussed.


