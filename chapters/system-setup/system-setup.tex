\chapter{System Setup}\label{ch:system-setup}

The \gls{sdh}~\cite{shadow-dex-hand} is a sophisticated robotic hand with a wide range of sensory feedback capabilities. To develop and test algorithms for this complex system, simulation is an invaluable tool. In this chapter, the practical setup of the project is presented, which involves simulating the \gls{sdh} within the dynamic simulators, Gazebo~\cite{gazebo} and MuJoCo~\cite{todorov2012mujoco}. \medskip

This simulation is based on the Shadow Robot Company's~\cite{shadow-robotics} \gls{ros}~\cite{ros} packages~\cite{shadow-ros-packages}, which provide a flexible and customizable framework for controlling the hand. The first simulator Gazebo, a popular robot simulation environment, simulates the physics of the hand and its interaction with the environment. To ensure reproducibility and portability, the development framework which encapsulates the simulation environment is built within a Docker~\cite{docker} container, which allows for easily distributing the code and dependencies to other researchers. The second simulator MuJoCo, like Gazebo, is a dynamic simulator, but with a greater emphasis on accurate physics simulations, making it ideal for research involving phenomena such as friction. MuJoCo has seen applications in \gls{ml} applications such as \gls{rl}-based dexterous manipulation~\cite{dexterous-object-manipulation, charlesworth2021solving, dexmv:-imitation-learning-for-dexterous-manipulation-from-human-videos}. The MuJoCo physics engine comes in a conda virtual environment, which like Docker, enables reproducibility and portability. \medskip

Simulating the system provides the added benefit of readily accessible \gls{gt} values, ensuring accurate analysis and evaluation of algorithms. Moreover, safety concerns are mitigated in the simulation as the risk of damaging physical hardware or incurring additional costs for equipment and sensors is eliminated. Instead, these aspects can be readily simulated through virtual sensor inputs and virtual components. \medskip

In this chapter, a detailed description of the software architecture, simulated hardware and software used in this project is presented. This description separates what software is provided as well as produced, including \gls{ros} packages.

\section{Simulation Setup}\label{sec:system-setup-simulation-setup}

The \gls{cad} model of the \gls{sdh} is a highly detailed and accurate representation of the physical hand. The model is based on the original design of the real-world hand, which was developed by the Shadow Robot Company. The \gls{cad} model includes precise geometry for all of the hand's components, including the finger joints, tendons, and simple tactile sensors. The model also includes detailed material properties for each component, which are used to simulate the hand's physical behavior in the simulation.\medskip

The \gls{sdh}'s joints are labeled based on the joint location according to~\appref{app:joint-abbreviations} and placement from the fingertip i.e. the outermost joint on the middle finger is labeled \texttt{MF1} while the innermost is \texttt{MF4}. The wrist joints are labeled \texttt{WR1} and \texttt{WR2}, where \texttt{WR1} is responsible for the hand's yaw and \texttt{WR2} is responsible for the hand's pitch rotation. \medskip

The structure of the right hand can be seen in \figref{fig:robot-hand-skeleton} compared to a human hand in \figref{fig:human-hand-skeleton}. Here each joint is marked with a color, labeling the \gls{dof} for that particular joint. The red labels refer to joints with one \gls{dof}, blue refers to joints with a \gls{dof} equal to two and purple refers to two joints that are coupled. The coupling between joints is such that, a flexion of joint two imposes a constraint on joint one. If joint two exceeds a certain angle, it enforces joint one to flex accordingly to maintain the established constraint. In the robot hand, both coupled joints are mechanically linked to a single motor. \medskip

As seen here the \gls{sdh} provides human-like dexterity due to its similar kinematics, see~\figref{app:human-and-robot-hand-kinematics}, and the comparable \gls{rom} of each joint, see~\tabref{app:range-of-motion-shadow-hand} and~\tabref{app:range-of-motion-human-hand}. \medskip

\begin{figure}[h]
	\centering
	\begin{subfigure}[b]{0.48\textwidth}
		\centering
		\includegraphics[height=\textwidth]{chapters/system-setup/fig/robot-hand-skeleton-coupled.pdf}
		\caption{\gls{sdh} with joints color coded depending on the \gls{dof}. The total number of controllable joints can here be seen as \num{24}. This figure is based on~\cite{svg-robot-hand}.}
		\label{fig:robot-hand-skeleton}
	\end{subfigure}
	\hfill
	\begin{subfigure}[b]{0.48\textwidth}
		\centering
		\includegraphics[height=\textwidth]{chapters/system-setup/fig/human-skeleton-hand-coupled.pdf}
		\caption{Human hand with joints color coded depending on the \gls{dof}. The total number of controllable joints can here be seen as \num{25}~\cite{design-and-development-of-a-bilateral-therapeutic-hand-device-for-stroke-rehabilitation}. This figure is based on~\cite{svg-skeleton-hand}.}
		\label{fig:human-hand-skeleton}
	\end{subfigure}
	\caption{The \gls{sdh} and a human hand, red here marks a joint with one \gls{dof}, blue marks a joint with two, and purple marks coupled joints.}
	\label{fig:hands-dof}
\end{figure}

The hand's geometry is modeled using a combination of standard shapes and custom-designed components. For example, the finger joints are modeled using a series of cylinders and spheres, which are connected by virtual tendons to simulate the motion of the real-world hand. The tactile sensors on the simulated hand, at the writing of this project, are purely aesthetical as representative simulated tactile sensors are yet to be supported as a standard component of the Shadow Robotics development environment. To generate representative tactile data additional software is therefore required. The tactile sensors can be seen in \figref{fig:robot-hand-skeleton} as the ellipsoids mounted at each fingertip. \medskip

Multiple hand configurations are available including a left hand, right hand, and both configurations mounted on \gls{ur} manipulators~\cite{shadow-hand-configurations}. The configuration chosen for this project is a Shadow Dexterous right hand without being mounted on a manipulator.~\figref{fig:simulated-robot-hand} shows the \gls{cad} model of the \gls{sdh} in simulation.

\begin{figure}[h]
	\begin{small}
		\begin{center}
			\includegraphics[width=0.5\textwidth]{chapters/system-setup/fig/simulation-robot-hand.png}
		\end{center}
		\caption{A cutout of the simulated \gls{sdh}.}
		\label{fig:simulated-robot-hand}
	\end{small}
\end{figure}
\newpage
\section{Software Setup}\label{sec:system-setup-software-setup}

The software for this project consists of two parts: the software provided and the software produced.

\subsection{Provided Software}\label{sec:system-setup-simulation-setup-provided}

The Gazebo simulation environment is shipped in a custom Ubuntu-based docker container~\cite{docker, ubuntu-docker-image} with the necessary libraries to communicate and develop applications on the simulated as well as the physical hand, wrapped within a catkin workspace~\cite{catkin}. Additionally, the container comes with common-use libraries for Python and C++ development in \gls{ros} including \texttt{numpy}\cite{numpy}, \texttt{OpenCV}~\cite{opencv} and \texttt{dynamic\_reconfiguration}~\cite{dynamic-reconfiguration}. A simplified overview of the development environment can be seen illustrated in \figref{fig:package-diagram}. The communication between the provided \gls{ros} packages and the Gazebo simulator is achieved through the \gls{ros}-Gazebo framework. 

\begin{figure}[!h]
	\begin{small}
		\begin{center}
			\includegraphics[width=0.5\textwidth]{chapters/system-setup/fig/init-package-diagram.pdf}
		\end{center}
		\caption{The boxes marked with grey are \gls{ros} packages, while the white are modules.}
		\label{fig:package-diagram}
	\end{small}
\end{figure}

When communicating with the \gls{sdh}, the primary interface provided is the hand commander i.e. \texttt{SrHandCommander} which enables functionalities such as retrieving the current state of the hand, executing given path plans etc. To plan and execute a high-resolution path \mvar{\mat{Q}_{\text{full}}\inR{m\times 24}}, where \mvar{m} is the number of joint configurations and \num{24} is the number of joints in the hand, a sequence of waypoints \mat{Q} form a low-resolution path of \mvar{\vec{q}_i = \rvec{q_0,\vsep q_1,\vsep \dots,\vsep q_n}^\T\inR{24}} where \mvar{i \in \{0,1,\dots,m_w\}} with \mvar{m_w} being the number of waypoints. \mat{Q} can thus be written as
\begin{equation}\label{eq:waypoint-path-matrix}
	\mat{Q} = 
		\left[\begin{array}{ccc}
			\vec{q}_1^\T \\
			\vec{q}_1^\T \\
			\vdots \\
			\vec{q}_{m_w}^\T
		\end{array}\right] \inR{m_w \times 24}.
\end{equation}

\texttt{SrHandCommander} parses the path to the \texttt{move\_group} handled by MoveIt~\cite{reducing-the-barrier-to-entry-of-complex-robotic-software:-a-moveit!-case-study}. After receiving the plan, MoveIt first checks the validity of the start, goal and intermediate configurations concerning the robot's joint limits, collision constraints, and other constraints like self-collision avoidance. This helps to ensure that the planned path is feasible and safe for the robot to execute. Once validated, the path \mvar{\mat{Q}_{\text{valid}}\inR{m_w\times 24}} is parsed to \gls{ompl}~\cite{the-open-motion-planning-library} where a safe high-resolution path is built using some chosen sample-based single- or multi-query path planning method to build the full high-resolution path \mvar{\mat{Q}_{\text{full}}\inR{m\times 24}}. Some examples of these methods include \gls{prm}, \gls{fmt} and \gls{rrt}-Connect, the last of which is the default used in the provided software and the one chosen for this project. To execute the path the development environment provides \texttt{SrJointPositionController} i.e. a joint space position controller by default, which is the one chosen for this project. \medskip

The controller is built using the \texttt{ros\_control}~\cite{ros-control} framework. The \texttt{ros\_control} package provides a hardware-agnostic interface, to enable the same controllers on multiple different robotics platforms, along with the same controller being applicable on real hardware as well as simulated hardware. Finally, the hardware interface communicates with \num{20} \gls{pid} controllers, one for each independent joint to ensure control. The software architecture for communicating with the robotic hand can be seen in \figref{fig:hand-communication-architecture}, which was inspired by figures from~\cite{shadow-robotics-control-description,shadow-robotics-firmware,shadow-robotics-controlling-the-hand}.
\begin{figure}[!h]
		\begin{center}
			\includegraphics[width=\textwidth]{chapters/system-setup/fig/system-communication-scheme.pdf}
		\end{center}
		\caption{Diagram showing the communication and control flow of interacting with the \gls{sdh}.}
		\label{fig:hand-communication-architecture}
\end{figure}

When executing an example path,~\figref{fig:coupling-and-planning-graph} shows joint angles during execution on the simulated hand throughout \SI{15}{\second} using the \gls{rrt}-Connect planner. The path here consists of five waypoints for \texttt{FFJ2} and \texttt{FFJ3}, whereas none is set for \texttt{FFJ1}. This is to demonstrate the coupling as \texttt{FFJ1} still follows the motion of \texttt{FFj2} even though no motion is commanded. \medskip

\newpage

\begin{figure}[h]
		\begin{center}
			\includegraphics[width=0.6\textwidth]{chapters/system-setup/fig/q-angles.pdf}
		\end{center}
		\caption{Showing the joint trajectory of the first finger's joints \texttt{FFJ2} and \texttt{FFJ3}, where the coupling between \texttt{FFJ1} and \texttt{FFJ2} is shown.}
		\label{fig:coupling-and-planning-graph}
\end{figure}

The MuJoCo~\cite{todorov2012mujoco} simulation runs a \gls{ah}~\cite{learning-complex-dexterous-manipulation-with-deep-reinforcement-learning-and-demonstrations}, which is a \gls{sdh} equipped with \num{30} controllable parameters, \num{24} for the hand's joints and \num{6} for the hand's position \mvar{\vec{p}_{hnd}=\rvec{p_x,p_y,p_z}} and orientation \mvar{\vec{u}=\rvec{\phi,\theta,\psi}} in \gls{rpy} format. The system stack for MuJoCo is similar to that of Gazebo and is therefore not repeated. \medskip

% These make up the agent's actions \mvar{a}, while the states {s} also include the position of the prop when performing the relocation task.

\subsection{Produced Software}\label{sec:system-setup-simulation-setup-produced}

To execute this project, software was developed to communicate with and extract data from the simulated \gls{sdh}. The software is structured into two \gls{ros} packages: \texttt{shadow\_hand} and \texttt{sr\_tactile\_perception}, with an additional utility package \texttt{ros\_utils}. \medskip

The \texttt{shadow\_hand} package is a software wrapper structured semantically in a similar way as the hand itself, as a \texttt{ShadowHand} object contains five \texttt{ShadowFinger}, each labeled as one of \texttt{TH}, \texttt{FF}, \texttt{MF}, \texttt{RF} or \texttt{LF} and a \texttt{ShadowWrist}~\ref{app:joint-abbreviations}. Each finger is equipped with a reader that samples the tactile data provided by the simulator's physics engine at \SI{100}{\hertz}. This wrapper is written as an easy-to-use interface to the hand which provides reliable bookkeeping of sensor data. The structure can be seen illustrated in~\figref{fig:shadow-hand}. \medskip

The \texttt{sr\_tactile\_perception} package is an interface for logging, processing and visualizing the tactile data using the \texttt{shadow\_hand}. Additionally, this package is responsible for managing and executing experiments. This can be seen as the top diagram in~\figref{fig:tactile-perception-package}. \medskip

Both of these packages are contained within a \texttt{in\_hand\_pose\_estimation} meta package which additionally contain custom tools for easy launching and execution of experiments with live plotting of tactile data. Additionally, each of these tools provides a high degree of easy-to-access flexibility in terms of model and configuration loading.\medskip

The last package is \texttt{ros\_utils} which contains important utilities used in the project such as type transformations, logging, data conversion, live plotting, encoding, decoding etc. This package contains both Python and C++ tools and can be found in~\cite{ros-utils-repo}.\medskip

The software developed in this project can be found in~\cite{in-hand-pose-estimation-repo}.


% \begin{figure}[!h]
% 	\begin{center}
% 		\includegraphics[width=0.6\textwidth]{chapters/system-setup/fig/produced-software-diagram.pdf}
% 	\end{center}
% 	\caption{Produced software architecture showing the wrapper and API structure.}
% 	\label{fig:produced-software}
% \end{figure}
% \begin{figure}[!h]
% 	\begin{center}
% 		\includegraphics[width=0.6\textwidth]{chapters/system-setup/fig/tactile-perception-package.pdf}
% 	\end{center}
% 	\caption{Produced software architecture showing the wrapper and API structure.}
% 	\label{fig:tactile-perception-package}
% \end{figure}


\begin{figure}[!h]
	\centering
	\begin{subfigure}[b]{0.48\textwidth}
		\centering
		\includegraphics[width=\textwidth]{chapters/system-setup/fig/produced-software-diagram.pdf}
		\caption{\texttt{shadow\_hand} package hierarchy structure and communication scheme.}
		\label{fig:shadow-hand}
	\end{subfigure}
	\hfill
	\begin{subfigure}[b]{0.48\textwidth}
		\centering
		\includegraphics[width=\textwidth]{chapters/system-setup/fig/tactile-perception-package.pdf}
		\caption{\texttt{sr\_tactile\_perception} package with functionalities and \texttt{ros\_utils} with common nice-to-have utilities.}
		\label{fig:tactile-perception-package}
	\end{subfigure}
	\caption{Developed \gls{ros} packages for executing this project.}
	\label{fig:shadow-hand-tactile-perception-package}
\end{figure}