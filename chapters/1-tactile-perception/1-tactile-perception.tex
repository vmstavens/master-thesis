\chapter{Tactile Perception} \label{ch:1-tactile-perception}

The field of robotics has been advancing rapidly in recent years, and tactile sensing is an important aspect of this advancement. Tactile data is vital for robots to interact with their environment and perform tasks with precision. One way to generate such data is through simulation, and regression neural networks have shown promise in simulating realistic tactile data. In this chapter, we analyze the performance of a regression neural network in simulating realistic tactile data, with a specific focus on its ability to simulate contact normals and skew forces. Contact normals are important for accurately estimating the pose of an object in contact with a robot, while skew forces are crucial for predicting the behavior of an object when it is grasped and manipulated by a robot's end-effector. We describe the technique behind the neural network and its architecture, and present our methodology for testing the network. Our methodology involves testing the network with a variety of input data and analyzing the output for accuracy and realism. We present our findings and discuss the strengths and weaknesses of the neural network in simulating tactile data, and draw conclusions about its overall performance and potential for future development. This analysis will provide valuable insights for researchers and practitioners working in the field of robotics and will pave the way for future advancements in tactile sensing and manipulation.

\section{Introduction} \label{sec:1-tactile-perception-introduction}

\begin{itemize}
	\item start with performance specifications regarding the two problems the tactile perception should be able to solve: can the are the points representative, according to the source yes, but are the normals also representative. This is important for the pose estimation problem. When manipulating the object the finger needs to be able to exert enough friction to be able to hold the object, is this true. compute the contact friction and see if it is stronger than the gravity on the subject. Use numbers to verify if the performance is acceptable
	\item 
\end{itemize}


\section{Method}\label{sec:1-tactile-perception-method}


\section{Experimental Setup}\label{sec:1-tactile-perception-experimental-setup}


\section{Results}\label{sec:1-tactile-perception-results}


\section{Discussion \& Conclusion}\label{sec:1-tactile-perception-discussion-and-conclusion}



% \section{Related Work} \label{sec:1-tactile-perception-related-work}

% Here we cite the related work by \texttt{\textbackslash cite\{source-label\}} like this \cite{recent-progress-in-technologies-for-tactile-sensors}