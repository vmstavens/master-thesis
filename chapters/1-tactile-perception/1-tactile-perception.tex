\chapter{Tactile Perception} \label{ch:1-tactile-perception}

\section{Introduction} \label{sec:1-tactile-perception-introduction}

\begin{itemize}
	\item start with performance specifications regarding the two problems the tactile perception should be able to solve: can the are the points representative, according to the source yes, but are the normals also representative. This is important for the pose estimation problem. When manipulating the object the finger needs to be able to exert enough friction to be able to hold the object, is this true. compute the contact friction and see if it is stronger than the gravity on the subject. Use numbers to verify if the performance is acceptable
	\item 
\end{itemize}


\section{Method}\label{sec:1-tactile-perception-method}


\section{Experimental Setup}\label{sec:1-tactile-perception-experimental-setup}


\section{Results}\label{sec:1-tactile-perception-results}


\section{Discussion \& Conclusion}\label{sec:1-tactile-perception-discussion-and-conclusion}



% \section{Related Work} \label{sec:1-tactile-perception-related-work}

% Here we cite the related work by \texttt{\textbackslash cite\{source-label\}} like this \cite{recent-progress-in-technologies-for-tactile-sensors}