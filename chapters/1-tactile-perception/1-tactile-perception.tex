\chapter{Tactile Perception} \label{ch:1-tactile-perception}

This chapter presents an analysis of the performance of a regression neural network in simulating realistic tactile data, with a specific focus on its ability to simulate contact normals and skew forces. Contact normals are essential for accurately estimating the pose of an object in contact, while skew forces are critical for predicting the behavior of an object when it is grasped and manipulated by the Shadow Dexterous hand. \medskip

The technique behind the neural network, including its architecture, is described, and the methodology used to test the network is presented. The testing methodology involves the use of various input data, and the output is analyzed for accuracy and realism. The findings are presented and discussed, including the strengths and weaknesses of the network in simulating tactile data. Finally, an assessment is made of the network's ability to produce tactile data that is realistic. \medskip

The software used in this project is a regression neural network implemented as a Gazebo \texttt{ModelPlugin}~\cite{gazebo-model-plugin} in C++. However, the \gls{dl} model plugin used in the original publication~\cite{simulation-of-the-syntouch-biotac-sensor} has not been updated since 2018, making the code incompatible with the current version of Gazebo API. Moreover, the licensing issues with the files in the \texttt{xmlrpc++} library, which were used for base64 encoding and decoding, necessitated their removal~\cite{base64-encoding-decoding-licensing-issue}. To address these issues, each has been resolved and the plugin has been reorganized and repackaged for compatibility with the current version of Gazebo. The original version of the plugin can be found in~\cite{ruppel-philipp-biotac-gazebo-plugin}, while the fixed and updated version is available at~\cite{melbye-staven-biotac-sim-plugin}. \medskip

The availability of the updated plugin ensures that the project can continue to benefit from the capabilities of the \gls{mlp} based \gls{dl} model for simulating realistic tactile data in the current version of Gazebo.


\section{Method}\label{sec:1-tactile-perception-method}

\subsection{Network Architecture}\label{sec:1-tactile-perception-method-network-architecture}

\subsection{Network Training Procedure}\label{sec:1-tactile-perception-method-network-training-procedure}


network architecture:
 - in 13 temp, force position



neural network architecture
training method
what makes it special

[figure of NN architecture]

\section{Experimental Setup}\label{sec:1-tactile-perception-experimental-setup}

picture of hand touching flat, edge, and sphere

ground truth vectors presented on figures and how errors were computed

holding bunny, and logging forces. Forces compared to the theoretical ideal.


\section{Results}\label{sec:1-tactile-perception-results}

show torques, forces and normals in grid plot.

normal cone and mirrroring error. How it was solved


What is the error.

\section{Discussion \& Conclusion}\label{sec:1-tactile-perception-discussion-and-conclusion}

It seems to represent the real phenomenons realistically 


% \section{Related Work} \label{sec:1-tactile-perception-related-work}

% Here we cite the related work by \texttt{\textbackslash cite\{source-label\}} like this \cite{recent-progress-in-technologies-for-tactile-sensors}