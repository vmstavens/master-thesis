\chapter{Introduction} \label{ch:intro}

Subject matter terms are addressed with \texttt{\textbackslash gls\{glossary-label\}} like so \gls{glossary-label}. \\
Acronyms are addressed with either with their long equivalent \texttt{\textbackslash acrlong\{gls-label\}} which gives \acrlong{acronym-label}
or the short equivalent \texttt{\textbackslash acrshort\{gls-label\}} which gives \acrshort{acronym-label}. \\
Subject matter terms can also be multi structure \texttt{\textbackslash gls\{glossary-multi-label\}} which gives \gls{glossary-multi-label} (see terms and acronyms above). \medskip

The developments in robotics as a field has over the past years provided automation solutions to execute repetitive manual tasks with high efficiency and reliability \fakecite. One of the most common tasks being pick and place tasks which involves picking un an object from one position and placing is in another. This is can be parted into the following subparts: Object localization, pose estimation, grasping and placing. In the solutions currently present for industrial use \gls{cv} is used for object localization and \gls{pe} due to the low cost of cameras and the fields maturity. However, while these solutions may be sufficient for certain tasks they fundamentally suffer from the weaknesses introduced by vision techniques. These include a great number of outliers caused by occlusions, reflecting, transparent or homogeneous surfaces, and repetitive structures when solving the \gls{corr-problem}. These problems as of the writing of this project have jet to be completely solved. In industrial settings transparency and especially reflectance become relevant since metallic parts 
tend to appear frequently and have this high reflectance \fakecite. With regards to transparency the rise of \gls{dl} has in present time proven its worth and provides proof of concept solutions for narrow cases in pose estimation of transparent objects \cite{6dof-pose-estimation-of-transparent-object-from-a-single-rgb-d-image}.




% Robot engineering replacing manual labor\\
% Manual labor of different industries (farming, health, transport), focus on Factory work \\
% Bin picking as a generic problem and its sub parts \\
% Current solutions, their parts with pros and cons (localization, pose estimation, grasping, placing) (deeplearning on transparent objects) \\
% How is this project going to solve the problems present in the current solutions \\
% What problems are going to be addressed, how are they going to be addressed and what is the solution's subparts \\
