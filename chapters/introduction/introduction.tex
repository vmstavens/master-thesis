\chapter{Introduction}\label{ch:intro}

% Subject matter terms are addressed with \texttt{\textbackslash gls\{glossary-label\}} like so \gls{glossary-label}. \\
% Acronyms are addressed with either with their long equivalent \texttt{\textbackslash acrlong\{gls-label\}} which gives \acrlong{acronym-label}
% or the short equivalent \texttt{\textbackslash acrshort\{gls-label\}} which gives \acrshort{acronym-label}. \\
% Subject matter terms can also be multi structure \texttt{\textbackslash gls\{glossary-multi-label\}} which gives \gls{glossary-multi-label} (see terms and acronyms above). \medskip

\section{Context}\label{sec:intro-context}

% historical context
As of 2022 humanity has developed tools for unprecedented growth in wealth and technology on a global scale. In such times a great deal of consumerism and interconnection is present with people needing product produced faster and more consistently than ever. 
% this is referred to as industry 4.0
As one would expect, this creates a high demand for manufacturers to reliably and consistently being able to provide products, while also remaining flexible, as the demand for different product change rapidly. 
% argue for why automation is better
% https://www.fishmancorp.com/robotics-manufacturing/
% https://www.universal-robots.com/blog/solving-complex-problems-with-innovative-concepts-and-robotic-solutions/
In order to provide great volumes of products, manual labor has is essential as assembly, transport and manipulation processes rely on these. Due to these types of manual labor being largely done by unskilled workers, automation alternatives are being adopted which provides benefits. 
% benefits for employer
These include for the employer: Avoid having to pay monthly salaries to unskilled labored individuals doing manual tasks, here the automation solution only requires electrical energy and potential supervision by a qualified individual. Potential risks are also involved when hiring humans as the workforce can be inconsistent due to human error\fakecite or left out due to illness etc. Considerations with regards workers rights such as working conditions and wage also needs not to be considered. These cause production limitations in the form of stand still hours, such as bathroom and lunch breaks along with after work hours and holidays. 
% benefits for the employee 
This replacement of manual labor also benefits the employee, as boring and physically wearing work is automated, enabling the employees to take on different and less wearing roles. While the issue of labor unemployment becomes apparent solutions which provide support to already hired workers have been developed, such as \gls{cobot}\fakecite. \medskip
% categories of problems in robotics for factories
When implementing automation of production lines using robotics, certain categories of problems are revealed. These include: Assembly, alteration and pick-and-place, the last being the one of interest in this project. 
\section{Problem Description}\label{sec:intro-problem-description}
% what does a pnp robot consist of?
For a robot to be capable of performing pick-and-place operations tasks it must consists of the following parts: The robot manipulator which is the robot arm, an end effector being the hand of the robot used to interact or manipulate objects of interest, sensors to obtain data from the environment and provide context to solve the given task and controllers to control the robot's motion.
% applications
Pick and place robots are used in a wide variety of different fields such as 
sorting of waste \cite*{robotic-pick-and-toss-facilitates-urban-waste-sorting}
handling of food \cite*{automation-of-mobile-pick-and-place-robotic-system-for-small-food-industry}\cite*{development-of-a-food-handling-soft-robot-hand-considering-a-high-speed-pick-and-place-task} and factory bin picking \cite*{real-time-industrial-bin-picking-with-a-hybrid-deep-learning-engineering-approach} \cite*{a-bin-picking-benchmark-for-systematic-evaluation-of-robotic-pick-and-place-systems} \cite{generic-development-of-bin-pick-and-place-system-based-on-robot-operating-system}. The solutions in these industries are examples of subcategories under the pick and place problem, namely sorting and bin picking. Since both of these are sub categories of the pick and place problem, they fundamentally follow the same sequential four phases from start to end. 
% All of these problems fundamentally contain the same structure as shared among all pick-and-place problems.
These steps being pre-grasping, grasping, transport, and placement \cite*{a-bin-picking-benchmark-for-systematic-evaluation-of-robotic-pick-and-place-systems}.
% pre-grasp
The pre-grasp phase involves localizing the object(s), potentially estimating their pose and executing the trajectory in order to move the end effectors grasp, collision free to said object(s). Here different potential grasp can be considered in order to determine the best pose for the end effector
% grasping

% transport

% placement


\section{Thesis Overview}\label{sec:intro-thesis-overview}































































% The developments in robotics as a field has over the past years provided automation solutions to execute repetitive manual tasks with high efficiency and reliability \fakecite. One of the most common tasks being pick and place tasks which involves picking un an object from one position and placing is in another. This is can be parted into the following subparts: Object localization, pose estimation, grasping and placing. In the solutions currently present for industrial use \gls{cv} is used for object localization and \gls{pe} due to the low cost of cameras and the fields maturity. However, while these solutions may be sufficient for certain tasks they fundamentally suffer from the weaknesses introduced by vision techniques. These include a great number of outliers caused by occlusions, reflecting, transparent or homogeneous surfaces, and repetitive structures when solving the \gls{corr-problem}. These problems as of the writing of this project have jet to be completely solved. Promising results have been found with the rise of \gls{dl} which in present time has proven its versatility and provides proof of concept solutions for narrow cases in pose estimation of transparent \cite{6dof-pose-estimation-of-transparent-object-from-a-single-rgb-d-image} and reflective \medskip

% \begin{minipage}{0.45\textwidth}
% 	objects \cite{6d-pose-estimation-of-objects:-recent-technologies-and-challenges}. This is relevant since industrial settings often contain transparent and especially reflecting objects as metallic parts tend to appear frequently and have high reflectances. To solve these problems this project aims to perform in-hand pose estimation through only the use of tactile sensors. Specifically this will be done on a Shadow Dexterous Hand \cite{shadow-dex-hand} with 20 \gls{dof}. Using tactile inputs rather than visual, eliminates the weaknesses mentioned above. A schematic showing the hand can be seen in \figref{fig:shadow-dex-hand-schematic}. Using this approach, the overall problem can be partitioned into 3 sub-problems labeled problem 1, 2 and 3. Problem 1 involves modeling the contact between the gripper's fingers and the object, also referred to as tactile perception. Problem 2 is to convert the collected data from problem 1 to meaningful surface data, treat these data as features and use them to estimate pose candidates. Finally problem 3 involves in-hand manipulation, such that further information is gained by probing the object. Here new desired surface points are found through intelligent probing such that strong surface features are found to better identify the object's correct pose. \medskip
% 	\end{minipage} 
% 	\hfill
% 	\begin{minipage}{0.45\textwidth}
% 	\begin{figure}[H]
% 		\begin{small}
% 			\begin{center}
% 				\includegraphics[width=0.95\textwidth]{chapters/introduction/fig/shadow-dex-hand-vector.pdf}
% 			\end{center}
% 			\caption{Schematic of Shadow Dexterous Hand from Shadow Robots, based on \cite{shadow-dex-hand-schematic}. The measurements are in \SI{}{\milli\metre}.}
% 			\label{fig:shadow-dex-hand-schematic}
% 		\end{small}
% 	\end{figure}
% \end{minipage}

% Thus the hypothesis of this projects $\text{H}_1$, will be testing if intelligent probing for strong features increases in-hand pose estimation performance, with the null hypothesis $\text{H}_0$ being that there is no statistical significant difference in the pose estimation performance of the system if the probing is done randomly or intelligently at a certainty level of \highlight{95\%}. Here pose estimation performance is quantified in terms of mean execution time for estimating the pose with an accuracy greater than \highlight{95\%}. \medskip

% The development of this project is done in the \gls{docker} provided by Shadow Robotics for simulation, control and development of the hand \cite{shadow-dex-github}. Here a hardware-simulation agnostic \gls{ros} \cite{ros} control \cite{ros-control} interface is found, which contains fundamental tools to interact with the robot hand. The dynamic simulation environment Gazebo \cite{gazebo} is likewise packaged as part of the \gls{docker} and is thus the one used for this project.
% To solve the problems presented, the \gls{ros} packages in \tabref{tab:software-package-table} will be applied, where \texttt{ros\_utils} and \texttt{in\_hand\_pose\_estimation} will be developed during this project.

% \begin{table}[h]
% 	\begin{small}
% 		\begin{center}
% 			\begin{tabular}[c]{ | l r | l | } \hline
% 				\cellcolor{tableheader} \textbf{Package}           & \cellcolor{tableheader} & \multicolumn{1}{l|}{\cellcolor{tableheader} \textbf{Description}} \\ \hline \hline
% 				\texttt{in\_hand\_pose\_estimation}                & \meta{meta} & \textbf{Project package of the in-hand pose estimation system} \\ \hline
% 				\hspace{0.3cm} \texttt{in\_hand\_pose\_estimation} &             & Integration of the full in-hand pose estimation pipeline  \\ \hline
% 				\hspace{0.3cm} \texttt{sr\_tactile\_image}         & \pkg{pkg}   & Extraction of tactile perception  \\ \hline
% 				\hspace{0.3cm} \texttt{sr\_pose\_estiamtion}       & \pkg{pkg}   & Estimate the pose of object based on tactile perception \\ \hline
% 				\hspace{0.3cm} \texttt{sr\_hand\_manipulation}     & \pkg{pkg}   & Manipulate object in hand to probe for strong features \\ \hline \hline
% 				\texttt{sr\_common}                                & \meta{meta} & \textbf{Shadow package for commonly used tools} \\ \hline
% 				\hspace{0.3cm} \texttt{sr\_common}                 &             & Implements commonly used tools such as messages \\ \hline
% 				\hspace{0.3cm} \texttt{sr\_robot\_msgs}            & \pkg{pkg}   & Messages used to communicate with the robot hand  \\ \hline 
% 				\hspace{0.3cm} \texttt{\dots}                      &             &  \\ \hline \hline
% 				\texttt{sr\_core}                                  & \meta{meta} & \textbf{Shadow package for core tools} \\ \hline
% 				\hspace{0.3cm} \texttt{sr\_core}                   &             & Implements core features of the hand such as hardware interfacing \\ \hline
% 				\hspace{0.3cm} \texttt{sr\_hand}                   & \pkg{pkg}   & Contains the hand commander for controlling the robot hand  \\ \hline
% 				\hspace{0.3cm} \texttt{\dots}                      &             &  \\ \hline \hline
% 				\texttt{ros\_utils}                                & \pkg{pkg}   & \textbf{Utilities for interfacing ROS/Gazebo/MoveIt/Eigen etc} \\ \hline
% 			\end{tabular}
% 		\end{center}
% 		\caption{Software packages used in the in-hand pose estimation system.}
% 		\label{tab:software-package-table}
% 	\end{small}
% \end{table}

% To present this work, the \gls{sota} solutions to each of the three problems described above will be presented in \chapref{ch:state-of-the-art}, where the best fitting methods for this use case will be chosen. In \chapref{ch:1-tactile-perception} to \chapref{ch:3-in-hand-manipulation} these solutions will be presented, analyzed, and their performance discussed and concluded upon. In \chapref{ch:4-system-integration} the system integration will be presented and the total performance of the system will be concluded. Finally in \chapref{ch:discussion} and \chapref{ch:conclusion} the results and methods will be discussed with potential improvement for future iterations and the project til be concluded.