\chapter{Discussion}\label{ch:discussion}


Overall Discussion and Conclusions:

The project consisted of three main chapters that explored different aspects of the research topic. In the tactile perception estimation chapter, the focus was on estimating contact positions, contact normals, and skew forces. Successful estimation was achieved using simulated contact points and the grasping matrix obtained from a previous study. This approach proved effective in estimating contact points accurately. The estimation of contact normals was accomplished by utilizing estimated linear velocities from simulated contact points and applying the \gls{rls} method. The results obtained on flat surfaces demonstrated low error rates and were considered satisfactory. However, due to time constraints, contact normals were not estimated on surfaces other than flat ones, which limited the validation of the technique's applicability. In these cases, the Gazebo physics engine provided acceptable results and served as a substitute for normal estimation. The performance of Gazebo's physics engine was deemed satisfactory. Estimating contact normals offers a significant advantage as it enables the identification of robust surface features. This, in turn, minimizes the search space for the pose estimation algorithm discussed in a previous chapter. It was observed that more complex surface features, such as corners, resulted in higher angle errors, which aligned with initial expectations.

The pose estimation chapter focused on estimating the pose of a known object, specifically the Stanford bunny, using synthetic and sampled source data. The problem was effectively formalized for both cases, and the core methodologies, namely \gls{gnc} and \gls{rcqp}, were presented along with the experimental setups. Synthetic source data proved to be successful in estimating the object's pose, meeting the predefined success criteria of angle errors below \SI{5}{\degree} and positional errors below \SI{1}{cm}. The performance was robust even with an outlier ratio of \SI{10}{\percent}. However, when dealing with sampled data, the methods faced challenges due to the sparsity of discernible features, which hindered the identification of sufficient feature correspondences. This limitation indicated the need for intelligent probing techniques to enhance feature detection. The execution times and number of iterations were consistent with the trends observed in the original paper. Additionally, a weight convergence analysis revealed an expected trend of convergence towards zero and one, but further investigation is required to fully understand the specific values at which the weights converge. Despite these challenges, the \gls{gnc}+\gls{rcqp} method exhibited promising performance in pose estimation for synthetic data, highlighting its potential for robust and accurate estimation.

In the dexterous manipulation chapter, the focus shifted to the \gls{dapg} algorithm and its capabilities for learning from demonstrations in the context of dexterous manipulation tasks, specifically the relocation task. The findings provided strong evidence that the \gls{dapg} algorithm successfully learned from demonstrations and demonstrated its capability to perform the relocation task. The trained algorithm exhibited promising results, and the outcomes followed a distribution that resembled a normal distribution. However, upon closer analysis, it was discovered that the assumption of homoscedasticity, which assumes equal variances across the performance values, was not strictly adhered to. This departure from homoscedasticity indicated the presence of variability or heterogeneity in the algorithm's performance across different relocation tasks. Additional investigation was conducted to compare the underlying distributions, which revealed a statistically significant difference between the agent trained by the authors and the alternative approaches. This significant difference demonstrated that the agent trained by the authors outperformed the alternatives in terms of its ability to successfully accomplish the relocation task.

In summary, the project provided valuable insights into the limitations and potential of \gls{dl} models in tactile perception estimation, pose estimation, and dexterous manipulation. It highlighted the importance of considering alternative methods, such as physics engine simulations,

 to supplement or replace \gls{dl} models when realistic tactile information is challenging to simulate. Improving feature detection and data sampling techniques is crucial for enhancing the accuracy of pose estimation, particularly when working with sampled data. The \gls{dapg} algorithm demonstrated effective learning capabilities for dexterous manipulation tasks, although the presence of variability in performance across different tasks needs to be considered. These findings contribute to the ongoing development of robust and efficient algorithms for tactile perception and manipulation tasks, and future studies could explore different \gls{dl} architectures or combinations of methods to further improve accuracy and usefulness in estimating tactile perception.