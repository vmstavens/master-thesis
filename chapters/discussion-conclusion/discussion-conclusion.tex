\chapter{Discussion \& Conclusion}\label{ch:discussion-conlcusion}


This project set out to develop an alternative pipeline for the \gls{pnp} problem, which uses tactile inputs rather than visual. To design this pipeline, three subproblems were identified
%
\begin{enumerate}
	\item To model the contact between the gripper's tactile sensors and the object, also referred to as \gls{tp}, such that useful data can be extracted for pose estimation.
	\item To convert the collected data from problem \ref{prob:1} to estimated pose candidates.
	\item To perform in-hand manipulation.
\end{enumerate}

% A restatement of your research question
% % problem 1
% \begin{problem} \label{prob:1}
% 	\normalfont involves modeling the contact between the gripper's tactile sensors and the object, also referred to as \gls{tp}, such that useful data can be extracted for pose estimation.
% \end{problem}
% % problem 2
% \begin{problem} \label{prob:2}
% 	\normalfont is to convert the collected data from problem \ref{prob:1} to estimated pose candidates.
% \end{problem}
% % problem 3
% \begin{problem} \label{prob:3}
% 	\normalfont involves in-hand manipulation. Since the initial grasp of the object might not be oriented in a manner where the recognizable features make context with the tactile sensors, manipulating the object within the \gls{ee}'s grasp will enable further information gathering. Thus the final problem is to control the \gls{ee} in such a manner that the tactile sensors make contact with the object through manipulation.
% \end{problem}
% The system setup chapter provided a detailed description of the software and simulation setup used in the project. The simulation of the Shadow Dexterous Hand (SDH) using Gazebo and MuJoCo proved to be a valuable tool for developing and testing algorithms for this complex robotic hand. The chapter highlighted the practical aspects of the project setup, including the use of Docker containers for reproducibility and portability of the development framework. The CAD model of the SDH was described, emphasizing its highly detailed and accurate representation for realistic physics simulation.
% A summary of your key arguments and/or results
% A short discussion of the implications of your research
% The project consisted of three main chapters that explored different aspects of the research topic. \medskip

In the \gls{tp} estimation chapter, the focus was on estimating contact positions, contact normals, and skew forces. Successful estimation was achieved using simulated contact points and the grasping matrix obtained from Ruppel et al.~\cite{simulation-of-the-syntouch-biotac-sensor}. This approach proved effective in estimating contact points accurately. The estimation of contact normals was accomplished by utilizing estimated linear velocities from simulated contact points and applying the \gls{rls} method. The results obtained on flat surfaces demonstrated low error rates and were considered satisfactory. However, due to time constraints, contact normals were not estimated on surfaces other than flat ones, which limited the validation of the technique's applicability. In these cases, the Gazebo physics engine provided acceptable results and served as a substitute for normal estimation. The performance of Gazebo's physics engine was deemed satisfactory. Estimating contact normals offers a significant advantage as it enables the identification of robust surface features. This, in turn, minimizes the search space for the pose estimation algorithm discussed in a previous chapter. It was observed that more complex surface features, such as corners, resulted in higher angle errors, which aligned with initial expectations. \medskip

The findings in this chapter additionally highlight the limitations and challenges associated with simulating realistic tactile information using the chosen \gls{dl} model. The \gls{dl} model did not provide useful information for tactile force sensing due to inaccurate simulation of electrode activations. This emphasizes the need for careful evaluation of DL models in specific applications, as they may not always deliver the expected benefits. The project was unable to replicate the tactile information from the original paper, indicating possible reasons such as non-representative weights and potential differences in Gazebo's API. Retraining the \gls{dl} model with a legitimate dataset could address these issues, but it was not feasible within the project's time constraints. Instead, the project relied on the forces and torques generated by Gazebo's physics engine, which were considered satisfactory based on the presented force and torque cones. While the \gls{dl} model did not meet expectations, the project managed to find an alternative solution to approximate tactile information for the given constraints. \medskip

The \gls{pe} chapter focused on estimating the pose of a known object, specifically the Stanford bunny, using synthetic and sampled source data. The problem was effectively formalized for both cases and the core methodologies, namely \gls{gnc} and \gls{rcqp}, were presented along with the experimental setups. Synthetic source data proved to be successful in estimating the object's pose, meeting the predefined success criteria of angle errors below \SI{5}{\degree} and positional errors below \SI{1}{cm} with an outlier ratio of \SI{10}{\percent}. However, when dealing with sampled data, the methods faced challenges due to the sparsity of strong features, which hindered the identification of sufficient feature correspondences. This limitation indicated the need for intelligent probing techniques to enhance feature detection through a great pool of data. The execution times and number of iterations were consistent with the trends observed in the original paper. Additionally, a weight convergence analysis revealed an expected trend of convergence towards zero and one, but further investigation is required to fully understand the specific ratio of weights converging to zero or one. Despite these challenges, the \gls{gnc}+\gls{rcqp} method exhibited promising performance in \gls{pe} for synthetic data, highlighting its potential for robust and accurate estimation. \medskip

In the dexterous manipulation chapter, the focus shifted to the \gls{dapg} algorithm and its capabilities for learning from demonstrations in the context of dexterous manipulation tasks, specifically the relocation task. The findings provided strong evidence that the \gls{dapg} algorithm successfully learned from demonstrations and demonstrated its capability to perform the relocation task. The trained algorithm exhibited promising results, and the outcomes followed a distribution that resembled a normal distribution. However, upon closer analysis, it was discovered that the assumption of homoscedasticity, which assumes equal variances across the performance values, was not strictly adhered to. This departure from homoscedasticity indicated the presence of variability or heterogeneity in the algorithm's performance across different relocation tasks. Additional investigation was conducted to compare the underlying distributions, which revealed a statistically significant difference between the agent trained by the authors and the alternative approaches. This significant difference demonstrated that the agent trained by the authors outperformed the alternative in terms of its ability to successfully accomplish the relocation task. \medskip

In summary, the project provided valuable insights into the limitations and potential of \gls{dl} models in tactile perception, \gls{pe}, and dexterous manipulation. It highlighted the importance of considering alternative methods, such as physics engine simulations, to supplement or replace \gls{dl} models when realistic tactile information is challenging to simulate. Improving feature detection and data sampling techniques is crucial for enhancing the accuracy of \gls{pe}, particularly when working with sampled data. The \gls{dapg} algorithm demonstrated effective learning capabilities for dexterous manipulation tasks, although the presence of variability in performance across different randomization seeds needs to be considered. \medskip

These findings contribute to the ongoing development of robust and efficient algorithms for tactile perception and manipulation tasks. Future studies could explore different \gls{dl} architectures or combinations of methods to further improve accuracy and usefulness in estimating tactile perception. One potential approach is to combine the solutions for problem~\ref{prob:1} and problem~\ref{prob:2} using time-dependent state estimation methods like the Kalman filter, which would involve continuously collecting tactile sensor data and estimating the object pose in a loop. By incorporating the time dependency of the system, this approach would have the potential to enhance \gls{pe} convergence and optimize computation time for real-time performance.
% In future iterations of this project, a potential approach is to combine the solutions for problem~\ref{prob:1} and problem~\ref{prob:2} using time-dependent state estimation methods like the Kalman Filter. This combination would involve continuously collecting tactile sensor data and estimating the object pose in a loop. By incorporating the time dependency of the system, this approach has the potential to enhance \gls{pe} convergence and optimize computation time for real-time performance.