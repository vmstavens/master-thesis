
\chapter{State of the Art} \label{ch:state-of-the-art}

\section{Problem 1 - Tactile Perception} \label{sec:lit-rev-problem-1}

Based on the contact model described in \chapref{ch:modeling}


% in order to perform tactile perception we need to model the contact between the end effector and the object of interest -> 
% 	This model needs to provide descriptions of parameters such as pressure and area for the contact to be 
% 	This model needs to contain the pressure and deformation caused by the contact in order to recreate the surface of contact





% What are contact models and what do they describe?
For tactile perception it is needed to model the contact between the robotic manipulator finger and the object. The model will here 

% we need a model


%  what do we wish to model contact forces and how they can relate to deformation. Friction to determine how much force to apply in order to keep the object in the grasp.


%  


% flush out what exactly the model i want to use looks like


When considering different methods for modeling contact interfaces, each can be categorized depending on which parameters the model describe the relation between. This results in the following groupings: contact-area-force models, stress-strain models, force-displacement models and 

% How are models grouped and what groupings exist?

% Contact models can be grouped based on the what parameters the relation is describing.







% Contact Area vs. Applied Force


% Hertzian contact model

% Soft Contact Model (More general Hertzian model)

% Viscoelastic Soft Contact Model.

% 	 Kelvin–Voigt/Maxwell model
% 	  Fung’s model

% Other contact models (research)



% Boussinesq–Cerruti’s

% Love’s solution 




Love's formulation 




What is tactile perception? Why is it relevant? \\
How is a tactile sensor constructed \cite{recent-progress-in-technologies-for-tactile-sensors}
what different types exist and which one is present in the model provided.


\textit{"Representations of tactile data are commonly either inspired by machine vision feature descriptors"}

often used in computer vision context, where each tactile image 

Addressing the problem 





\section{Problem 2 - Pose Estimation} \label{sec:lit-rev-problem-2}

\section{Problem 3 - In-Hand Manipulation} \label{sec:lit-rev-problem-3}