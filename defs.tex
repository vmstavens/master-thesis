% Language
\usepackage[T1]{fontenc}
\usepackage[utf8]{inputenc}
\usepackage[english]{babel}

% Page formatting
\usepackage{geometry}
\geometry{a4paper, margin=25mm, includefoot}

% Font, line height, paragraph indentation
% \usepackage{mathptmx} % Times New Roman
\usepackage{newtxtext} % Times New Roman (Modern + Math symbols)
\usepackage[varvw]{newtxmath} % Times New Roman (Modern + Math symbols)

\linespread{1.2}
\setlength{\parindent}{0pt}
% \usepackage{setspace}

% Chapter/section title formatting
\usepackage{titlesec}
\titleformat{\chapter}[display]{\normalfont\Large}{\chaptertitlename\ \thechapter}{0pt}{\huge\bfseries}[\vspace{8pt}\titlerule]
\titlespacing{\chapter}{0cm}{0cm}{0.5cm}

\titleformat{\paragraph}
{\normalfont\normalsize\bfseries}{\theparagraph}{1em}{}
\titlespacing*{\paragraph}
{0pt}{3.25ex plus 1ex minus .2ex}{1.5ex plus .2ex}

% Multi-column TOC
\usepackage{multicol}
% \makeatletter
% \renewcommand{\tableofcontents}[1][\contentsname]{%
%     \chapter*{#1}
%     \begin{small}
%     % \setlength{\columnseprule}{0.5pt}
%     \setlength{\columnsep}{20pt}
%     \begin{multicols}{2}
%         \@starttoc{toc}
%     \end{multicols}
%     \end{small}
% }
% \makeatother

% Used for moving abstract from the center of the page to the top
\usepackage{etoolbox}
\patchcmd{\abstract}{\null\vfil}{}{}{}

% Figures and captions
% \usepackage{caption}
\usepackage{float}
\usepackage{graphicx}
\usepackage{wrapfig}
\usepackage[font=small,labelfont=bf]{caption}
\usepackage[labelformat=simple]{subcaption}
\renewcommand\thesubfigure{(\alph{subfigure})}

\def \FigureAbbreviaition {Fig.}
\newcommand{\figref}[1]{\FigureAbbreviaition\ \ref{#1}}
\newcommand{\tabref}[1]{Table \ref{#1}}
\newcommand{\chapref}[1]{Chapter \ref{#1}}

% Figure caption formatting
\AtBeginDocument{%
\captionsetup[figure]{name={\FigureAbbreviaition},aboveskip=10pt,belowskip=-10pt}
\captionsetup[subfigure]{aboveskip=10pt,belowskip=0pt}
\captionsetup[table]{aboveskip=10pt,belowskip=-10pt}
% \captionsetup[subtable]{aboveskip=10pt,belowskip=0pt}
}



% Array
\usepackage{array}
\usepackage{textpos}

% Math and symbols
% \usepackage{amsmath, amssymb, bm} % newtxmath has same defines as amssymb

\usepackage{amsmath}
\newtheorem{problem}{Problem}

% \theoremstyle{remark}

\usepackage{amsmath, bm}
\usepackage{mathtools}
\usepackage{verbatim}
\usepackage{xfrac}
\usepackage{tensor}
\usepackage{interval}
\usepackage{commath} % for \abs, \norm
\usepackage{physics} % for qty(), qty{} etc. (automatic parentheses)

% Custom commands
\usepackage{xstring}
\usepackage{xspace}

\newcommand{\fakecite}[0]{\hl{\textbackslash cite}\xspace}
\renewcommand{\vec}[1]{\ensuremath{\bm{#1}}} % vector [amsmath, bm]

\newcommand{\mat}[1]{\ensuremath{\bm{\mathrm{#1}}}} % matrix [amsmath, bm]
\newcommand{\T}[0]{\top} % transpose ^T
\newcommand{\inv}[0]{\ensuremath{^{-1}}} % inverse ^{-1}
\newcommand{\pinv}[0]{\ensuremath{^{\dagger}}} % pseudo-inverse ^{-1}

\newcommand{\rvec}[1]{\ensuremath{\renewcommand{\arraystretch}{0.6}\begin{bmatrix} #1 \end{bmatrix}}}
\newcommand{\rlist}[1]{\ensuremath{\renewcommand{\arraystretch}{0.6}\begin{Bmatrix} #1 \end{Bmatrix}}}
\newcommand{\cvec}[1]{\ensuremath{\renewcommand{\arraystretch}{1.0}\begin{bmatrix} #1 \end{bmatrix}}}

\newcommand{\twodots}{\mathinner {\ldotp \ldotp}} % .. (two dots)
\renewcommand{\secref}[1]{\hyperref[#1]{\ref*{#1}\ \nameref*{#1}}}
% \newcommand{\tf}[3][T]{\ensuremath{\tensor[^{#2}]{\mat{#1}}{_{#3}}}}
\newcommand{\tf}[3][T]{\ensuremath{{\mat{#1}^{#2}_{#3}}}}
\newcommand{\highlight}[1]{\hl{#1}\xspace}
\newcommand*\of{\qty} % $f\of(x)$
\newcommand{\R}[0]{\mathbb{R}} % real number R

\newcommand{\mvar}[1]{$#1$} % environment for mathematical variable $f(x)$ => \mvar{f(x)}
\newcommand{\robframe}[1]{\mvar{\{\text{#1}\}}}
\newcommand{\inR}[1]{ \in\R^{#1} }

\newcommand{\Csf}{\mathcal{E}_{f,\text{SF}} }
\newcommand{\Chf}{\mathcal{C}_{f,\text{HF}} }

\newcommand{\vsep}{\;} % vector separator
\newcommand{\mfor}{\text{ for }} % for in equation
\newcommand{\mand}{\text{ and }} % and in equation
\newcommand{\mspa}{\qquad} % for in equation


% usage: \tf{from}{to} | \tf[T]{from}{to} | \tf[R]{from}{to} | \tf[t]{from}{to} etc.
% \newcommand{\tf}[3][T]{%
%     \IfEqCase{#1}{%
%         {T}{\ensuremath{\tensor[^{#2}]{\mat{#1}}{_{#3}}}}%
%         {R}{\ensuremath{\tensor[^{#2}]{\mat{#1}}{_{#3}}}}%
%         {t}{\ensuremath{\tensor[^{#2}]{\vec{#1}}{_{#3}}}}%
%         {p}{\ensuremath{\tensor[^{#2}]{\vec{#1}}{_{#3}}}}%
%     }[\PackageError{tf}{Undefined option: #1}{}]%
% }%

% Equation spacing
\AtBeginDocument{%
\abovedisplayskip=6pt plus 2pt minus 2pt
\abovedisplayshortskip=6pt plus 2pt minus 2pt
\belowdisplayskip=6pt plus 2pt minus 2pt
\belowdisplayshortskip=6pt plus 2pt minus 2pt
}

% SI units + config and custom units
\usepackage{siunitx}[=v2]
\sisetup{
    per-mode=fraction, fraction-function=\sfrac, % fractions
    % round-mode=figures, round-precision=3, % rounding
    output-exponent-marker=\ensuremath{\mathrm{e}},
    separate-uncertainty=true, multi-part-units=single, % for \SI{2 \pm 0.2}{\rad},
    binary-units=true, % for \byte, \giga etc.
}
\DeclareSIUnit \pixel {px}

% Text highlight (\hl)
\usepackage{soul}

% Enumeration and tables
\usepackage{enumerate}
\usepackage{enumitem}
\usepackage{multirow} 
\usepackage{multicol}
\usepackage{ltablex}
\usepackage{spreadtab}
\usepackage{booktabs}
\usepackage{tabto}
\usepackage{makecell}
\usepackage{diagbox}
\usepackage{tabularx}
\usepackage[export]{adjustbox}

% Custom tables commands for fixed-column sizes
\newcommand{\PreserveBackslash}[1]{\let\temp=\\#1\let\\=\temp}
\newcolumntype{C}[1]{>{\PreserveBackslash\centering}p{#1}}
\newcolumntype{R}[1]{>{\PreserveBackslash\raggedleft}p{#1}}
\newcolumntype{L}[1]{>{\PreserveBackslash\raggedright}p{#1}}

\setlength\doublerulesep{0.3cm} % when using a double line, make extra space

% Enumeration formatting
\newcommand{\tabitem}{~~\llap{\textbullet}~~}
\newcommand{\sqrbulletsml}{\textcolor{black}{\raisebox{.45ex}{\rule{.6ex}{.6ex}}}}
\newcommand{\sqrbulletmed}{\textcolor{black}{\raisebox{.40ex}{\rule{.7ex}{.7ex}}}}

\renewcommand{\labelitemi}{\sqrbulletmed}
\renewcommand{\labelitemii}{\sqrbulletsml}
\renewcommand{\labelitemiii}{\sqrbulletsml}
\renewcommand{\labelitemiv}{\sqrbulletsml}

% Colors (latexcolor.com)
\usepackage[table]{xcolor}
\definecolor{cerulean}{rgb}{0.0, 0.48, 0.65}
\definecolor{earthyellow}{rgb}{0.88, 0.66, 0.37}
\definecolor{darkmagenta}{rgb}{0.55, 0.0, 0.55}
\definecolor{darkolivegreen}{rgb}{0.33, 0.42, 0.18}
\definecolor{codegray}{gray}{0.9}
\definecolor{gainsboro}{rgb}{0.86, 0.86, 0.86}
\definecolor{cyan}{rgb}{0.0, 1.0, 1.0}

\definecolor{light-yellow}{RGB}{255, 255, 137}
\definecolor{light-blue}{RGB}{148, 201, 233}

\definecolor{tableheader}{gray}{0.9}

\definecolor{metaorange}{rgb}{0.98, 0.54, 0.13} % also known as flame orange

% colored square box
% https://tex.stackexchange.com/questions/201300/inline-boxes-alternative-to-pifonts-non-filled-but-shadowed-box
\newcommand{\sqbox}[1]{\textcolor{#1}{\rule{1.2ex}{1.2ex}}}

% Code snippets
\usepackage{listings}
% \renewcommand{\arraystretch}{1.2}
\renewcommand{\tabcolsep}{0.2cm}

% Code snippets formatting
\lstset{
    backgroundcolor=\color{black!5},        % set background color
    basicstyle=\footnotesize\ttfamily,      % basic font setting
    captionpos=b,                           % caption position
    frame=single,                           % draw a frame at the top and bottom of the code block
    framesep=5pt,                           % frame margin
    xleftmargin=5pt,                        % frame margin
    xrightmargin=5pt,                       % frame margin
    tabsize=4,                              % tab space width
    showstringspaces=false,                 % don't mark spaces in strings
    breaklines=true,                        % wrap lines
    commentstyle=\color{darkolivegreen},    % comment color
    keywordstyle=\color{darkmagenta},       % keyword color
    stringstyle=\color{earthyellow},        % string color
    identifierstyle=\color{cerulean},
}

% Custom inline-code command (\code)
%\newcommand{\lstinln}[1]{\colorbox{gainsboro}{\texttt{#1}}}
\newcommand{\code}[1]{%
  \begingroup\setlength{\fboxsep}{2pt}%
  \colorbox{gainsboro}{\texttt{\hspace{2pt}\vphantom{Ay}#1\hspace{2pt}}}%
  \endgroup
}

% package labels
\usepackage{tikz}
\newcommand{\meta}[1]{ \tikz[baseline=(X.base)]\node [draw=metaorange,fill=metaorange,semithick,rectangle,inner sep=2.5pt, rounded corners=2pt] (X) {  \textbf{\textcolor{white}{\textsf{#1}}}}; \hphantom}
\newcommand{\pkg}[1] { \tikz[baseline=(X.base)]\node [draw=blue,fill=blue,semithick,rectangle,inner sep=1.3pt, rounded corners=2pt] (X)             {  \textbf{\textcolor{white}{\textsf{#1}}}}; \hphantom}

% Appendices
\usepackage[toc]{appendix}

% Bibliography and citation
\usepackage[style=numeric, sorting=none]{biblatex}
\usepackage{csquotes}
\DeclareNameAlias{sortname}{family-given}
\DeclareNameAlias{default}{family-given}

% resources
\addbibresource{resources/resources-appendix.bib}
\addbibresource{resources/resources-tactile-perception.bib}
\addbibresource{resources/resources-system-setup.bib}
\addbibresource{resources/resources-words.bib}
\addbibresource{resources/resources-introduction.bib}
\addbibresource{resources/resources-modeling.bib}
\addbibresource{resources/resources-sota-tactile-perception.bib}
\addbibresource{resources/resources-sota-pose-estimation.bib}
\addbibresource{resources/resources-sota-in-hand-manipulation.bib}

% Section cross-referencing
\usepackage{nameref}
\usepackage{hyperref}



% Label counter formatting (e.g. Figure 1 or Figure 2.1)
% In case you want the numbering to start with the chapter number e.g. 1.1 1.2 1.3 for chapter 1 figures, use this instead of the one below
\usepackage{chngcntr} 
\AtBeginDocument{%
    \counterwithin{figure}{chapter}
    \counterwithin{table}{chapter}
    \counterwithin{equation}{chapter}
    \counterwithin{lstlisting}{chapter}
}

\usepackage{tcolorbox}
\tcbuselibrary{theorems}

% In case you want the numbering to continue between chapters e.g. 1 2 3 4 5, use this instead of the one above
% \usepackage{chngcntr} 
% \AtBeginDocument{%
%     \counterwithout{figure}{chapter}
%     \counterwithout{table}{chapter}
%     \counterwithout{equation}{chapter}
%     \counterwithout{lstlisting}{chapter}
% }

% Glossaries and acronyms
% must be loaded last + no empty includes in main document.
\usepackage[automake, acronym, nogroupskip, nonumberlist]{glossaries}
\usepackage{glossary-mcols}
\setglossarysection{section}

% command for dual entries
% \newdualentry[<options>]{<label>}{<abbrv>}{<long>}{<description>}
% https://tex.stackexchange.com/a/368666
\newcommand*{\newdualentry}[5][]{%
  \newglossaryentry{main-#2}{%
    name={#4 (\glslink{#2}{#3})},%
    text={#3\glsadd{#2}},%
    description={{#5}},%
    #1%
  }%
  \newglossaryentry{#2}{%
    type=\acronymtype,%
    first={\glslink{main-#2}{#4 (#3)}},%
    name={#3\glsadd{main-#2}},%
    description={\glslink{main-#2}{#4}},%
    plural={#2s},%
    firstplural={\glsentrydesc{#2}s (\glsentryplural{#2})}
  }%
}

% setup and load glossaries
\makeglossaries
\loadglsentries{glossary}

% only hyperlink first-time glossary entries
% \renewcommand*{\glslinkcheckfirsthyperhook}{%
%   \ifglsused{\glslabel}%
%   {%
%     \setkeys{glslink}{hyper=false}%
%   }%
%   {}%
% }